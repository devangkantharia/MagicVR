\section{Implementation}

The project is implemented in C++ using the libraries \BoostGeometry, \RxCpp, \rangeLib and \trajecmp.
The library \trajecmp is written by the author Michael Maier to detect gestures.
The process shown in~\autoref{fig:system-diagram} is implemented like this:

\begin{lstlisting}
const auto result_stream =
    compare(preprocess(input_stream),
            preprocess(pattern_stream));

result_stream.subscribe([](auto result) {
    // process result
});
\end{lstlisting}

\noindent
\inlineCpp{input\_stream} and \inlineCpp{pattern\_stream} are streams of trajectories.
Every time a input trajectory is emitted on \inlineCpp{input\_stream} it is preprocessed and compared with the (latest) preprocessed pattern trajectory of \inlineCpp{pattern\_stream}.
Usually, \inlineCpp{pattern\_stream} contains only one (static pattern) trajectory.
However, a pattern trajectory might be dynamically created.
For example, you could use another input trajectory as a pattern.\\

\noindent
You can compare the input trajectory with multiple patterns like this:

\begin{lstlisting}
const auto preprocessed_input_stream =
    preprocess(input_stream);

const auto result_1_stream =
    compare(preprocessed_input_stream,
            preprocess(pattern_1_stream));

const auto result_2_stream =
    compare(preprocessed_input_stream,
            preprocess(pattern_2_stream));

result_1_stream.subscribe([](auto result) {
    // process result
});

result_2_stream.subscribe([](auto result) {
    // process result
});
\end{lstlisting}

\noindent
All gestures of the project are implemented in the class \path{sources/magicvr/MagicTricks.cpp}.
Apart from the preprocessing steps shown in~\autoref{fig:preprocessing-steps} it contains a rotation transformation around the axis pointing upwards in the cave.
The event subscription is done in \path{sources/magicvr/AppController.cpp} and \path{sources/magicvr/AppControllerWithWandSupport.cpp}.